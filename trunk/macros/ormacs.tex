% ormacs.tex -- Oriya macros
%

\catcode`\@=11

%%%%% Font loading Macros

\font\or@VIII   = or8
\font\or@X      = or10
\font\or@XII    = or12
\font\or@XVII   = or17
\font\or@XXIV   = or17 at 24pt
\font\or@LXXII  = or17 at 72pt

\font\orsl@VIII = orsl8
\font\orsl@X    = orsl10
\font\orsl@XII  = orsl12
\font\orsl@XVII = orsl17

\font\orbf@VIII = orbf8
\font\orbf@X    = orbf10
\font\orbf@XII  = orbf12
\font\orbf@XVII = orbf17

\font\orbs@VIII = orbs8
\font\orbs@X    = orbs10
\font\orbs@XII  = orbs12
\font\orbs@XVII = orbs17

\font\orss@X	= orss10
\font\orss@XVII	= orss17
\font\orss@XXIV = orss17 at 24pt
\font\orss@LXXII= orss17 at 72pt

\font\orsssl@X	= orsssl10
\font\orssbf@X	= orssbf10
\font\orssbs@X	= orssbs10

\def\orVIII{\or@VIII\baselineskip=9pt}
\def\or{\or@X\baselineskip=12pt}
\def\orXII{\or@XII\baselineskip=15pt}
\def\orXVII{\or@XVII\baselineskip=20pt}
\def\orXXIV{\or@XXIV\baselineskip=26pt}
\def\orLXXII{\or@LXXII\baselineskip=72pt}

\def\orsl{\orsl@X\baselineskip=12pt}
\def\orbf{\orbf@X\baselineskip=12pt}
\def\orbs{\orbs@X\baselineskip=12pt}

\def\orss{\orss@X\baselineskip=12pt}
\def\orssXVII{\orss@XVII\baselineskip=20pt}
\def\orssXXIV{\orss@XXIV\baselineskip=26pt}
\def\orssLXXII{\orss@LXXII\baselineskip=72pt}

\def\orsssl{\orsssl@X\baselineskip=12pt}
\def\orssbf{\orssbf@X\baselineskip=12pt}
\def\orssbs{\orssbs@X\baselineskip=12pt}

%%%%% Oriya Characters

\def\<#1>{\char#1}

\input orcode

\newif\iformode\ormodefalse     % toggle for Oriya mode


%%%%% Syllable Composition Algorithm
%
% The intention here is to create macros that correctly parse ISCII input,
% and build the syllables on the fly. This will allow users to type their
% text in a ISCII editor, and then simply put it through TeX to view and
% print it. It will make processing a bit slow, but it saves a lot of 
% hassle with pre-processors.

% make the ISCII characters active, and let them call appropriate macros

% \IsciiCons is called by all consonants and full vowels, it will look 
% ahead to see what is next. The next character can be
%   a halant            call \IsciiHalant
%   a vowel sign        call \IsciiMatra
%   a modifier          call \IsciiModifier
%   something else      call \IsciiBuildSyllable

\def\IsciiCons#1{#1}

% \IsciiReph is called in the special case of the letter ra, it will look
% ahead to see what is next. The next character can be



% we have given all glyphs in our font symbolic names, we define them here.

\catcode`\@=12
\endinput
