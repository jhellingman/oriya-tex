% orman.tex -- user's manual of Oriya-TeX
% Copyright 1996, 1999 Jeroen Hellingman
%
% History:
%   20-JUN-1999 Added several conjuncts (JH)
%   21-MAY-1999 Added s+kha conjunct (JH)
%   30-MAY-1998 Added h+ma conjunct (JH)
%	15-MAY-1998 Fixed wrong vowel-sign ri (JH)
%   03-MAY-1998 Changed address, added vowel signs for rii, li, lii,
%               further minor modifications (JH)
%   07-SEP-1997 Added d+ga, dd+dda (JH)
%   17-MAY-1997 Several corrections and addition of table (JH)
%   11-MAY-1997 Added transcription table (JH)
%   02-MAR-1997 Small Edits (JH)
%   01-SEP-1996 Extended (JH)
%   15-AUG-1996 Created (Jeroen Hellingman)
%
%%%%% Macros

\input ormacs
\input 2kol
\setdoublecolumns{16cm}{24.6cm}{7.65cm}
\emergencystretch=30pt

\def\today{\number\day/\number\month/\number\year}

\rightheader={{\it Oriya-\TeX\ 1.0 Users' Manual}\hfill\folio}
\leftheader={\folio\hfill{\it Oriya-\TeX\ 1.0 Users' Manual}}

\voffset-1cm	% artifact of the 2kol macros

\def\ortex{Oriya-\TeX}
\def\obrace{\char"7B}
\def\cbrace{\char"7D}

\parindent=0pt

%%%%% Text

\beginsection \ortex

({\it printed on \today\/})\medskip

\beginsection Summary

This article gives an overview of the \ortex-package which can be
used in combination with \TeX\ for typesetting text in Oriya 
({\or uw[\ornukta aA}). \ortex\ can also be used in combination with other
packages, to typeset mixed language
documents. \ortex\ is part of a larger project to make all South Asian scripts
available for \TeX. The following South Asian scripts are now available
as Metafonts: Devanagari (for Sanskrit, Prakrit, Hindi, Marathi, Nepali,
Konkani), Gurumukhi (for Panjabi), Tamil, Malayalam, Telugu, Kannada, and
Singhalese. Some of these packages need some refinement and adaption to
La\TeX. The only important Indian scripts still missing are Gujarati and Bengali.

\beginsection License

This software is \copyright\ Copyright 1999 Jeroen Hellingman. However, it may
be freely distributed in the spirit of the GNU General Public License. The only
restriction is that you cannot place further restrictions on this or derived works
and you should give others you give this software or derived works everything, 
including the sources. Of course, works you print using this font are not derived 
works, but if you add or touch up a few characters, or create a post script font, 
based on this design, it is.

As I am a collector of aksharamala (first grade alphabet books, often very
nice and colourful), I will highly appreciate it if you send me one, especially
for Oriya, although all Indian scripts are welcome.

% This is still a work in progress. A beta version of the fonts is available
% for interested users. Note that the pre-processor is not yet available.
% 
% The following points are to be reviewed:
% (i)~Letters and symbols that might be missing.
% (ii)~Designs of specific letters might need improvements.
% (iii)~Spacing between letters can sometimes be incorrect.
% (iv)~There might be spelling mistakes in the samples.
% 
% Your help is requested for the following points:
% (i)~Better samples, I would prefer to have some pancatantra stories.
% (ii)~Resolving some technical details.
% (iii)~Converting the fonts to TrueType and PostScript type~1 fonts. (I don't have
% such fonts yet.)
% (iv)~Translating the manual to Oriya.
% 
% All suggestions and comments are welcome.

\bigskip

Jeroen Hellingman

Aletta Jacobsstraat 5

3404 XD IJsselstein

The Netherlands

telephone: +31 30 687 5444 (19:00--21:00 MET)

E-mail: {\tt<jehe@kabelfoon.nl>}

\vfill\eject

%\begindoublecolumns

\beginsection Introduction

The Oriya script is used for the Oriya language and a few minority languages, 
such as Khondi and Santali, spoken
in the state of Orissa in Eastern India. The Oriya language belongs to the
Indic branch of the large Indo-European language family and is
closely related to Bengali. It is spoken by about 30 million people, of them
roughly the half can read and write.

The Oriya script forms a link between the South and North Indian scripts. The
distinctive inner parts of some of the letters can be recognised when one knowns
Devanagari and
Bengali script, but the circular appearance reminds one of the curly South Indian
scripts. The explanation for all those circles is easy: traditionally the
script was written on palm-leaves with a sharp metal nib, and when one makes a
horizontal top-line like in Devanagari on a palm-leave, it will tear. To
avoid this, the top-lines where changed into the semi-circles which give
Oriya its characteristic appearance---described by Nakanishi\footnote*{Akira
Nakanishi, {\it Writing Systems of the World,} p. 54} as a parade of bald heads.

The Oriya alphabet follows the alphabetic order of Sanskrit. This
order is based on phonetic principles. The vowels come first, in pairs of 
a short and long sound, those articulated in the
back of the mouth first. After the `pure' vowels come the `diphtongs'
(in Sanskrit, {\it e} and
{\it o} are supposed to have been diphtongs originally).

\def\OB#1#2{$\hbox{\orXVII#1}\atop\hbox{\strut\it#2}$}

$$
\vbox{\halign{\it\hfill#\hfill\quad&&\it\hfill#\hfill\quad\cr
\OB{a}{a}	&\OB{aA}{\=a}	&\OB{i}{i}	&\OB{I}{\=\i}	&\OB{u}{u}	&\OB{U}{\=u}	\cr\noalign{\smallskip}
\OB{\orvowelri}{\d r}&\OB{\orvowelrii}{\d{\=r}}&\OB{\orvowelli}{\d l}&\OB{\orvowellii}{\d{\=l}}	\cr\noalign{\smallskip}
\OB{e}{e}	&\OB{E}{ai}	&\OB{o}{o}	&\OB{O}{au}	\cr
}}
$$

\noindent Three vowel modifiers traditionally follow the vowels. These are
{\or a/} {\it a\.m}, {\or aM} {\it a\d{m}}, and {\or aH} {\it a\d{h}},
which originally stood for a nasalisation, a following
nasal, and a following aspiration respectively. 
In Oriya, {\or H} {\it \d h} is now used to indicated that the following consonant
is doubled.
Here written on the letter {\or a} {\it a,}
they can appear on any letter. The modifiers are 
followed by the consonants arranged in rows (called {\it varg}) of related sounds. The rows are ordered
again by point of articulation, with semivowels and sibilants comming at the end, 
and in each row the voiceless sound first, followed by the aspirated, voiced,
voiced and aspirated sound and the corresponding nasal.

$$
\vbox{\halign{\it\hfill#\hfill\quad&&\it\hfill#\hfill\quad\cr
\OB{k}{ka}	&\OB{K}{kha}	&\OB{g}{ga}	&\OB{G}{gha}	&\OB{f}{\.na}	\cr\noalign{\smallskip}
\OB{c}{ca}	&\OB{C}{cha}	&\OB{j}{ja}	&\OB{J}{jha}	&\OB{F}{\~na}	\cr\noalign{\smallskip}
\OB{q}{\d ta}	&\OB{Q}{\d tha}	&\OB{w}{\d da}	&\OB{W}{\d dha}	&\OB{N}{\d na}	\cr\noalign{\smallskip}
\OB{t}{ta}	&\OB{T}{tha}	&\OB{d}{da}	&\OB{D}{dha}	&\OB{n}{na}	\cr\noalign{\smallskip}
\OB{p}{pa}	&\OB{P}{pha}	&\OB{b}{ba}	&\OB{B}{bha}	&\OB{m}{ma}	\cr\noalign{\smallskip}
\OB{Y}{ya}	&\OB{y}{\.ya}	&\OB{r}{ra}	&\OB{l}{la}	&\OB{v}{va}	\cr\noalign{\smallskip}
\OB{z}{\'sa}	&\OB{S}{\d sa}	&\OB{s}{sa}					\cr\noalign{\smallskip}
\OB{h}{ha}	&\OB{L}{\d la}	&	&\OB{w\ornukta}{\d{r}a}	&\OB{W\ornukta}{\d{r}ha}\cr
}}
$$

\noindent as shown in this table, all consonants are thought to be followed by the
short letter {\or a} {\it a.}\footnote{**}{The vowels {\or\orvowelrii, \orvowelli,}
and {\or\orvowellii} are not used in Oriya at all. They are borrowed from Sanskrit
(and even in Sanskrit the last one is only introduced for symmetry), and are
included in the alphabet for completeness. I've found only one reference to
the corresponding vowel signs for these three vowels. The letter {\or v} is a
recent invention to distinguish the sound {\it va} from {\it ba} in loanwords.
It is a combination of the vowel {\or o} with a secondary {\or b}.}

When a vowel follows a consonant, this is indicated by writing a special vowel sign,
attached to the consonant. These vowel signs sometimes even stand before the
consonant they apply to. Here the vowel signs are shown attached to the letter 
{\or g} {\it ga\/}:

$$
\vbox{\halign{\it\hfill#\hfill\quad&&\it\hfill#\hfill\quad\cr
\OB{g}{ga}	&\OB{gA}{g\=a}	&\OB{g[}{gi}	&\OB{gX}{g\=\i} &\OB{g]}{gu}	&\OB{gZ}{g\=u}	\cr\noalign{\smallskip}
\OB{g\orsignri}{g\d r}	&\OB{<g}{ge}	&\OB{<g>}{gai}	&\OB{<gA}{go} &\OB{<g*}{gau}	\cr\noalign{\smallskip}
}}
$$

Vowel signs often combine with the consonant they apply to:
The vowel sign for {\it\=a,} {\or\dotcircle A}, is often joined to the letter, as
in {\or kA} {\it k\=a.}
The vowel sign for {\it i,} {\or\dotcircle[}, `sinks' into the outer circle,
for example in 
{\or k[} {\it ki}, and the vowel sign for {\it u,} {\or\dotcircle]}, can be written
in one continuous stroke, as in {\or k]} {\it ku}. Sometimes
the vowel sign for {\it u,} {\or\dotcircle]}, becomes {\or\dotcircle\orsignuvar},
for example in {\or l]} {\it lu}.
On the letters  
{\or T} {\it tha\/} and {\or D} {\it dha}, the vowel sign for {\it i,}
{\or\dotcircle[}, becomes a little hook {\or\dotcircle\orsignivar}, written
below the letter: {\or T[} {\it thi}, {\or D[} {\it dhi}. However, recent
simplifications
of the script have done away with all these irregular vowel signs.

When no vowel follows a consonant at all, not even the short {\or a} {\it a,}
this can be indicated with
a sign called viram, {\or\dotcircle\orhalant}, for example, {\or k\orhalant} {\it k.}
However, this symbol is not
used that often. It is normally omitted at the end of words.

When two or more consonants follow each other, they are combined into a
consonant cluster or conjunct.
Often these conjuncts consist of a smaller version of the second letter subscribed
to the first letter, for example, {\or\orshca} {\it \'sca\/}
is used for {\or z\orhalant c}. Sometimes,
the outer circle of the subscribed consonant is omitted, as in {\or\orlka} {\it lka\/}
for {\or l\orhalant k}. In many cases, however, the original consonants can hardly
be recognised in the conjunct, for example in {\or\orkSa} {\it k\d sa\/}
for {\or k\orhalant S}. A fairly complete list of conjuncts is given as an appendix.

The letters {\or Y} {\it ya} and {\or r} {\it ra\/} are treated in a special way. 
When {\or r} {\it ra\/} is the first consonant of a cluster, it appears as the symbol
reph {\or\dotcircle\orreph} at
the end of the cluster, for example, {\or\orNNa\orreph} {\it r\d n\d na}. When it
is the last in a cluster, it appears as a subscribed hook, {\or\dotcircle\orsecra}, for
example in {\or p\orsecra} {\it pra.} When the letter {\or Y} {\it y\/} is the
last in a cluster,
it appears as the symbol {\or\dotcircle\orsecya}, for example, {\or k\orsecya} {\it
kya.} This same symbol {\or\dotcircle\orsecya} appears after the vowels {\or e} 
{\it e\/} and {\or o} {\it o\/} to modify their sounds.

The rules for combining consonants into conjuncts used to make typing and typesetting
Oriya a very complicated business.
Traditional Oriya typefaces in lead require several hundereds of different types.
A computer, however, can easily apply all the rules, and
make it possible to type only the base letters of Oriya in their phonetic order.

The spelling and the use of conjuncts is hardly standarized, and it is no exception
to come across spelling variations even inside a single book.

Punctuation in Oriya script follows the English practice, except that
the full stop is sometimes replaced by a standing bar {\or\ordanda}, called
{\it danda}, which is separated from the last word by a space to avoid
confusion with the vowel sign for {\or aA} {\it \=a}.

Finally, Oriya uses its own set of figures, {\or 1 2 3 4 5 6 7 8 9 0}, although
international figures are sometimes used as well.


\beginsection How the Oriya fonts were made

To the typeface designer, Oriya script poses
a big challenge: to create a font that, while it maintains the
characteristic Oriyan appearance, is still easily readable and also economises the
use of paper. To achieve this, the first most important issue is to find
the proper proportion between the size of the circle and the size of the
inner, distinctive features of each letter. Especially when drawing the
letters
in large size on a sheet of paper, one tends to make the circle far to
large, and, when the design is reduced to the normal size of a letter
the inner details have become so small as hardly visible. However,
the larger one makes the inner part, the more difficult it becomes
to draw a pleasing circle around it.

A second consideration
is the extend to which vowel signs and secondary characters are allowed to
stick below or above the letters, as this defines the necessary 
baseline distance, and also the texture of the printed page. Here, again,
one should not be tempted to make the vowel signs and secondary
characters to small, and one can often win much by using a ligature
instead of the standard combination---although the number of ligatures
that can be used is limited.

Because Oriya used to be written on palm-leaves, Oriya has little or no calligraphic
tradition. One cannot produce thick-thin transitions, and other niceties with
a metal nib. With the introduction of the printing press in India\footnote{*}{The first
printing in Oriya was probably done at the Serampore mission press in the early
nineteenth century. In 1809 this press printed a complete bible in Oriya.---B.S. Kesavan,
{\it History of Printing and Publishing in India,} Vol.~I, p.~254}, Oriya
typefaces with thick-thin transistions were produced. However, these
thick-thin transitions where not applied in a consistent way.

In this design, an attempt was made to add some calligraphic effects
to the letters. After some trial and error, a 30~degree cut nib was found
out to produce very acceptable results in most letters. Only in a few
letters some little tricks, like using a second, much thinner pen, where
required to make the result pleasing---notably in the diagonals
of {\or y} {\it \.ya}, {\or S} {\it \d{s}a}, and {\or x} {\it k\d{s}a}.

When I started the design of this font, I decided to look at some old Oriya
manuscripts first---which was easy, as the museum was just around the corner
from my home in Ahmedabad---and started to make skeleton studies. I sketched the
skeleton or `essence' of the letter 25 millimetres high on graph paper.
After this, I could easily read of the coordinates of the control points which
I then coded in Metafont. Now, by giving the pen somewhat more `body,' a more
usable letter would appear. Then, by changing the pen from a circle to a slanted 
ellipse, a more calligraphic appearance was achieved. The illustration
below demonstrates this for the letter {\or k} {\it ka\/}:

%\enddoublecolumns

\font\orskBIG=orsk10 at 47.62mm
\font\orBIG=or10 at 47.62mm
\font\orssBIG=orss10 at 47.62mm
\vskip1cm
\centerline{\orskBIG k\quad\orssBIG k\quad\orBIG k}
\vskip1cm

%\begindoublecolumns

On the left is the original skeleton, then comes the design drawn with
a circular nib, which I named {\orss kqk} {\it Cuttack}, after the (former) capital
of Orissa, as it is supposed to be modern, and on the right is the design drawm
with a elliptical nib, named {\or <kANAk\orreph} {\it Konark}, after the place
where the famous sun temple is located.

Of course, the designs didn't come out right the first time. Every letter
has gone through several iterations of the process of first a skeleton
sketch, then a large size picture on the screen, and finally a try-out
in running text on paper. The large circles can easily dominate the design, but the
distinctive
features of each letter need to be easily recognised. In my first designs, I~made
the inner features too small, which made reading text at small sizes difficult.

\beginsection How to Use Oriya-\TeX

A set of Oriya fonts alone doesn't make up an easy to use Oriya type-setting system.
You will also need an easy way of typing Oriya text. The complex
rules of Oriya script makes it cumbersome and error prone to type all
required shapes directly. A computer can handle the composition and allow
the user to type only the base characters in phonetic order. As the rules of
Oriya composition are not build into \TeX, a pre-processor will be used to accomplish
this. As the name suggests, this pre-processor needs to be run on your file
with Oriya text before \TeX\ can handle it.

The pre-processor will accept standard \TeX\ and La\TeX\ files, in which Oriya is typed
in an ASCII based transcription scheme. This scheme is based on the scheme used
by Frans Velthuis for Devanagari. The output is a file contains a lot
of gibberish which you don't want to look at, but it will tell \TeX\ how to
typeset your Oriya text. 

Before you can start, you will have to load some macros.
Users of plain \TeX\ have to input the file {\tt ormacs.tex} somewhere at the
start of their document.
Users of La\TeX\ can simply use the package {\tt oriya} to load the
required macros and fonts. \ortex\ supports the new font selection scheme.

To enable the pre-processor to recognise transcribed Oriya text, you will have to
enclose it in between the tags {\tt <OR>} and {\tt </OR>}. A special `dollar-mode' is
supplied for people who have to switch from Oriya to English constantly. When it
is used, Oriya can also be enclosed between dollar signs. In this case, math
has to be accessed with {\tt <MATH>} and {\tt </MATH>}.

The following table indicates how to type Oriya. The first collumn indicates the
transcription typed, the second the Oriya letter output, and the third the
scientific transcription. It should be noted that the scientific transcription
is based on the model of Sanskrit, and does not give an exact indication of the
actual pronunciation in Oriya.

\bigskip

\bgroup
\def\orstrut{\vtop to4pt{}\vbox to10pt{}}
\def\q{\quad\hfill}
\def\x{\vrule\ }
\def\xx{\vrule width.8pt}
\offinterlineskip
\halign{\orstrut\xx\ \tt#\q&\x#\q&\x#\q\xx\ &\tt#\q&\x#\q&\x#\q\xx\ &\tt#\q&\x#\q&\x#\q\xx\cr
\noalign{\hrule height.8pt}
a       & {\or a} or implicit       	& {\it a}         	&
ka      & {\or k}                  		& {\it ka}         	&
pa      & {\or p}                   	& {\it pa}        	\cr
aa      & {\or aA} or {\or\dotcircle A}	& {\it \=a}     	&
kha     & {\or K}                 		& {\it kha}	        &
pha, fa & {\or P}                   	& {\it pha, fa}     \cr
i       & {\or i} or {\or\dotcircle[}	& {\it i}       	&
ga      & {\or g}                   	& {\it ga}          &
ba, va  & {\or b}                  	 	& {\it ba, va}      \cr
ii      & {\or I} or {\or\dotcircle X}	& {\it \=\i}    	&
gha     & {\or G}                   	& {\it gha}         &
bha     & {\or B}                   	& {\it bha}         \cr
u       & {\or u} or {\or\dotcircle]}  	& {\it u}       	&
"na     & {\or f}                   	& {\it \.na}        &
ma      & {\or m}                   	& {\it ma}       	\cr
uu      & {\or U} or {\or\dotcircle Z}	& {\it \=u}         &
ca      & {\or c}                  		& {\it ca}	        &
ya      & {\or Y}                   	& {\it ya}          \cr
R       & {\or\orvowelri} or {\or\dotcircle\orsignri}	& {\it \d{r}}	&
cha     & {\or C}                 		& {\it cha}         &
.ya     & {\or y}                  		& {\it \.ya}        \cr
RR      & {\or\orvowelrii} or {\or\dotcircle\orsignrii}     		& {\it \d{\=r}}     &
ja      & {\or j}                       & {\it ja}          &
ra      & {\or r}                		& {\it ra}   		\cr
L       & {\or\orvowelli} or {\or\dotcircle\orsignli}               & {\it \d{l}}       &
jha     & {\or J}                 		& {\it jha}		    &
la 		& {\or l}            			& {\it la}          \cr
LL      & {\or\orvowellii} or {\or\dotcircle\orsignlii}              & {\it \d{\=l}}     &
\~{}na  & {\or F}                 		& {\it \~na}        &
.va     & {\or v}	                  	& {\it va}          \cr
e       & {\or e} or {\or<\dotcircle}	& {\it e}           &
.ta     & {\or q}                 		& {\it \d{t}a}      &
sha     & {\or z}                 		& {\it \'sa}        \cr
ai      & {\or E} or {\or<\dotcircle>}  & {\it ai}          &
.tha    & {\or Q}                		& {\it \d{t}ha}     &
.sa     & {\or S}                 		& {\it \d{s}a}      \cr
o       & {\or o} or {\or<\dotcircle A} & {\it o}           &
.da     & {\or w}                 		& {\it \d{d}a}      &
sa      & {\or s}                 		& {\it sa}          \cr
au      & {\or O} or {\or<\dotcircle*}  & {\it au}	        &
.dha    & {\or W}                		& {\it \d{d}ha}     &
ha      & {\or h}                  		& {\it ha}          \cr
/	    & {\or\dotcircle/}				& {\it \.m}			&
.na     & {\or N}                 		& {\it \d{n}a}      &
.la     & {\or L}                  		& {\it \d{l}a}      \cr
.m      & {\or\dotcircle M}          	& {\it \d{m}}       &
ta  	& {\or t}                  		& {\it ta}		    &
.ra     & {\or w\ornukta}               & {\it \d{r}a}	    \cr
.h      & {\or\dotcircle H}           	& {\it \d{h}}       &
tha     & {\or T}                 		& {\it tha}         &
.rha    & {\or W\ornukta}               & {\it \d{r}ha}     \cr
	    & 								& 					&
da      & {\or d}                  		& {\it da}          &
+		& {\or\dotcircle+} 				&					\cr
\~{}e   & {\or e\orsecya>}			 	& {\it \^e}        	&
dha     & {\or D}                 		& {\it dha}         &
		&								&					\cr
\~{}o 	& {\or o\orsecya>}				& {\it \^o}			&
na		& {\or n}                  		& {\it na}		    &
        &                       		&               	\cr
\noalign{\hrule height.8pt}
}\egroup

\bigskip

Note that {\or\orkSa} {\it k\d sa} and {\or\orjnya} {\it j\~na} are
conjuncts, and typed as {\tt k.sa} and {\tt j\~{}na} respectively.

When two vowels follow each other, care has to be taken that the pair can
be distinguished from a single vowel transcribed with two letters. This is
done by typing {\tt"} between them. For example, 
when a short {\or a} {\it a\/} follows another short {\or a} {\it a\/}. For
example
{\or kaN} {\it kaa\d{n},} should be typed as {\tt ka"a.n}. When there is
no danger for confusion, there is no need to separate vowels in this way. 
The following table should make all possible combinations clear:

\vfill\eject

\bgroup
\def\orstrut{\vtop to4pt{}\vbox to10pt{}}
\def\q{\quad\hfill}
\def\x{\vrule\ }
\def\xx{\vrule width.8pt}
\offinterlineskip
\halign{\orstrut\xx\ \tt#\q&\x#\q&\x#\q\xx\ &\tt#\q&\x#\q&\x#\q\xx\cr
\noalign{\hrule height.8pt}
aa		& {\or aA}		& {\it \=a}			&
kaa		& {\or kA}		& {\it k\=a}		\cr
a"a		& {\or aa}		& {\it aa}			&
ka"a	& {\or ka}		& {\it kaa}			\cr
a"aa	& {\or aaA}		& {\it a\=a}		&
ka"aa	& {\or kaA}		& {\it ka\=a}		\cr
ai		& {\or E}		& {\it ai}			&
kai		& {\or <k>}		& {\it kai}			\cr
a"i		& {\or ai}		& {\it a\"\i}		&
ka"i	& {\or ki}		& {\it ka\"\i}		\cr
au		& {\or O}		& {\it au}			&
kau		& {\or <k*}		& {\it kau}			\cr
a"u		& {\or au}		& {\it a\"u}		&
ka"u	& {\or ku}		& {\it ka\"u}		\cr
ii		& {\or I}		& {\it \=\i}		&
kii		& {\or kX}		& {\it k\=\i}		\cr
i"i		& {\or ii}		& {\it ii}			&
ki"i	& {\or k[i}		& {\it kii}			\cr
uu		& {\or U}		& {\it \=u}			&
kuu		& {\or kZ}		& {\it k\=u}		\cr
u"u		& {\or uu}		& {\it uu}			&
ku"u	& {\or k]u}		& {\it kuu}			\cr
\noalign{\hrule height.8pt}
}\egroup

\bigskip

It is important to take care to to type the inherent {\or a} {\it a\/} when it is 
silent between two letters. It may be omitted at the end of a word.

To control the exact appearance of conjuncts, you can use the
halant, which is transcribed as {\tt +}. When you want to use the
standard conjunct, just type the constituent consonants after each
other: for example, {\tt kra} gives you {\or\orkra} {\it kra.} If you want to
use an explicit halant, type {\tt k+ra} to get {\or k+r} {\it kra.} If you want
to use the variant {\or k\orsecra} {\it kra,} you can type it as {\tt k++ra}.

Consult the table in the appendix for the exact appearance of each conjunct and
possible variations.

In India,
there are several editors which allow phonetic order typing of Oriya script. Such editors
normally use the ISCII character set to represent Oriya characters internally.
To allow those people who have the luxery of an
ISCII editor to use Oriya-\TeX, the pre-processor can also process ISCII, or
convert between ISCII, Unicode, and transcription.

Oriya-\TeX\ can also be used to typeset Oriya in scientific transcription.
For this, the to be transcribed text has to be placed between {\tt<ORTR>} and
{\tt </ORTR>}. The transcription as shown in the tables will be used. A silent
{\it a\/} can be typed as {\tt *}, and a capital is produced by preceeding it
with a {\tt \^\ }. A whole sentence in capitals can be produced by putting it
in between {\tt \^\ \^\ }.


% TODO



%\enddoublecolumns

% 
% \beginsection Composition of Oriya
% 
% The composition algorithm for Oriya script accepts a sequence of Oriya characters,
% encoded following the Unicode standard, and delivers a sequence of composition
% instructions, which can be interpreted by a typesetting system, which is now \TeX, but
% can also be configured for PostScript or just plain glyph sequences which can
% be displayed on Windows.
% 
% For sake of the Oriya composition algorithm, we classify characters as follows:
% 
% $V$ full vowels, {\it i.e.,} {\or a} \dots\ {\or O}.
% 
% $C$ consonants, {\it i.e.,} {\or k} \dots\ {\or L}.
% 
% $M$ vowel signs or matras, {\it i.e.,} {\or\dotcircle A} \dots\ {\or <\dotcircle*}.
% 
% $D$ vowel modifiers, {\or\dotcircle/} candrabindu, {\or\dotcircle M}, anusvar, and
% {\or\dotcircle H} visarg.
% 
% $H$ the vowel omission sign viram, {\or\dotcircle\orhalant}.
% 
% $N$ the diacritic sign nukta, {\or\dotcircle\ornukta}.
% 
% $O$ any other character, for example punctuation marks, Roman letters, figures, and spaces.
% 
% The logic unit for composition is a syllable. For the sake of our composition algorithm,
% A syllable has the following structure:
% 
% syllable ::= cluster ending
% 
% cluster ::= letter halant letter
% 
% 
% The structure of a syllable has to be changed, following the set of rules given 
% below.
% 
% 1. re-ordering of the vowel-signs.
%    The parts of the vowel signs have to positioned at their proper position.
% 
%    vowel vowel	left	over	under		right
%    
%    a	a
%    \=a	aA
%    i	i
%    \=\i	
%    
%  
% 2. re-ordering of the reph
% 
% 	When the first consonant of a cluster is ra, then this ra is represent
% 	by the symbol reph, following the cluster and following consonants, but before
% 	a possible vowel modifier.
% 	
% 3. use of secondary consonants
% 
% 	when the last consonant of the cluster has a secondary version, this
% 	version is used, and placed in its proper position.
% 	
% 	consonant		left 	over	under	right
% 	
% 
% 4. apply consonant--consonant ligatures
% 5. apply consonant--vowel sign ligatures
% 6. apply vowel sign--modifier--reph ligatures
% 7. apply kerning.
% 
% 	all possible ligatures are given in the appendix.
% 
% 	
% 	
% 
% The output of the algorithm is a sequence of glyphs. Those glyphs can be classified
% as follows:

\vfill\eject

%%%%%%%%%%%%%%%%%%%%%%%%%%%%%%%%%%%%%%%%%%%%%%%%%%%%%%%%%%%%%%%%%%%%%%%%%%%%
% Sample page Konark
%%%%%%%%%%%%%%%%%%%%%%%%%%%%%%%%%%%%%%%%%%%%%%%%%%%%%%%%%%%%%%%%%%%%%%%%%%%%

\def\oralphabet{a aA i I u U \orvowelri\ e E o O a/ aM aH\par
k K g G f c C j J F q Q w W N t T d D n\par
p P b B m y Y r l L z S s h\par}
\def\ormatras{k kA k[ kX k] kZ k\orsignri\ <k <k> <kA <k* k/ kM kH\par}
\def\orfigures{1 2 3 4 5 6 7 8 9 0\par}
\def\orsample{\par nm\orska Ar. aAs\ornta]. k[m[t[ aC\ornta[?
aApN K]s[<r rh\ornta]. aApN\orngka\ sm\orsignri\orddha[ <hu.
aApN\orngka r nZaA bS\orreph\ m\orngga LmY <hU.
k'N KAi<b, cAhA nA kP[? 
aApN\orngka\ kAy\orsecya\orreph\orkra m k'N aC[?
Grk] aAs\ornta].
aApN\orngka] <dKA kr[ K]s[ <hl[.
aAuT<r m[z[bA.\par}

\font\ssbfXII=cmss10 at 12pt

\leftline{\ssbfXII Oriya}
\vskip2mm
\hrule
\vskip1cm
\leftline{Konark regular 72pt}
{\orLXXII a <kANAk\orreph

}

\vskip5mm
\leftline{Konark regular 24pt}

{
\orXXIV\oralphabet
\orXVII\ormatras
\orfigures
} 

\vskip3mm
\hrule
\vskip3mm

\begindoublecolumns

Konark normal 10 pt
{\or <kANAk\orreph\ <nAr\orsecma l \orsample}

\medskip Konark slanted 10 pt
{\orsl <kANAk\orreph\ s\orsecla\ornta w \orsample}

\medskip Konark bold 10 pt
{\orbf <kANAk\orreph\ <bAl\orhalant w \orsample}

\medskip Konark bold slanted 10 pt
{\orbs <kANAk\orreph\ <bAl\orhalant w s\orsecla\ornta w \orsample}

\enddoublecolumns

%%%%%%%%%%%%%%%%%%%%%%%%%%%%%%%%%%%%%%%%%%%%%%%%%%%%%%%%%%%%%%%%%%%%%%%%%%%%
% Sample page Cuttack
%%%%%%%%%%%%%%%%%%%%%%%%%%%%%%%%%%%%%%%%%%%%%%%%%%%%%%%%%%%%%%%%%%%%%%%%%%%%
\vfill\eject

\font\ssbfXII=cmss10 at 12pt

\leftline{\ssbfXII Oriya}
\vskip2mm
\hrule
\vskip1cm
\leftline{Cuttack regular 72pt}
{\orssLXXII a kqk

}

\vskip5mm
\leftline{Cuttack regular 24pt}

{\orssXXIV\oralphabet
\orssXVII\ormatras
\orfigures
} 

\vskip3mm
\hrule
\vskip3mm

\begindoublecolumns

Cuttack normal 10 pt
{\orss kqk <nAr\orsecma l \orsample}

\medskip Cuttack slanted 10 pt
{\orsssl kqk s\orsecla\ornta w \orsample}

\medskip Cuttack bold 10 pt
{\orssbf kqk <bAl\orhalant w \orsample}

\medskip Cuttack bold slanted 10 pt
{\orssbs kqk <bAl\orhalant w s\orsecla\ornta w \orsample}

\enddoublecolumns

%%%%%%%%%%%%%%%%%%%%%%%%%%%%%%%%%%%%%%%%%%%%%%%%%%%%%%%%%%%%%%%%%%%%%%%%%%%%%
% Tables
%%%%%%%%%%%%%%%%%%%%%%%%%%%%%%%%%%%%%%%%%%%%%%%%%%%%%%%%%%%%%%%%%%%%%%%%%%%%%
\vfill\eject

\leftline{\bf Table of Oriya letters}
\bigskip

All basic characters of the Oriya script are given, together with their Roman
transliteration, ISCII\footnote{*}{Indian Script Code for Information Interchange, 
IS 13194:1991} and Unicode\footnote{**}{The Unicode Standard, Version 1.0, Vol 1., 
and checked against the tables of version 2.014} code-position, ASCII representation, 
and name. In a few cases, two ISCII characters are needed to represent a single 
Oriya character. The vowel signs are given on a dotted circle, which represents 
the consonant or conjunct to which the vowel sign is to be applied.

Conjuncts and ligatures of consonants and vowel signs are given in the
next sections.

\medskip

\halign{\orXII #\quad\hfill&#\quad\hfill&\tt#\ \hfill&\tt#\ \hfill&\tt#\ \hfill&#\hfill\cr
\bf Oriya	&\bf Roman	&\bf ISCII	&\bf Unicode	&\bf ASCII	&\bf name	\cr\noalign{\medskip\leftline{\it vowels}\medskip}
%
a		&a		&A4		&0B05		&a	&vowel a 	\cr
aA		&\=a		&A5		&0B06		&aa	&vowel aa 	\cr
i		&i		&A6		&0B07		&i	&vowel i	\cr
I		&\=\i		&A7		&0B08		&ii	&vowel ii	\cr
u		&u		&A8		&0B09		&u	&vowel u	\cr
U		&\=u		&A9		&0B0A		&uu	&vowel uu	\cr
\orvowelri	&\d{r}		&AA		&0B0B		&R	&vowel ri	\cr
\orvowelrii	&\d{\=r}		&AA E9		&0B60		&RR	&vowel rii	\cr
\orvowelli	&\d{l}		&A6 E9		&0B0C		&L	&vowel li	\cr
\orvowellii	&\d{\=l}		&A7 E9 		&0B61		&LL	&vowel lii	\cr
e		&e		&AC		&0B0F		&e	&vowel e	\cr
E		&ai		&AD		&0B10		&ai	&vowel ai	\cr
o		&o		&B0		&0B13		&o	&vowel o	\cr
O		&au		&B1		&0B14		&au	&vowel au	\cr\noalign{\medskip\leftline{\it vowel modifiers}\medskip}
%
a/		&a\.m		&A1		&0B01		&/	&candrabindu	\cr
aM		&a\d{m}		&A2		&0B02		&.m	&anusvar	\cr
aH		&a\d{h}		&A3		&0B03		&.h	&visarg		\cr\noalign{\medskip\leftline{\it consonants}\medskip}
%
k		&ka		&B3		&0B15		&k	&consonant ka	\cr
K		&kha		&B4		&0B16		&kh	&consonant kha	\cr
g		&ga		&B5		&0B17		&g	&consonant ga	\cr
G		&gha		&B6		&0B18		&gh	&consonant gha	\cr
f		&\.na		&B7		&0B19		&"n	&consonant nga	\cr\noalign{\medskip}
%
c		&ca		&B8		&0B1A		&c	&consonant ca	\cr
C		&cha		&B9		&0B1B		&ch	&consonant cha	\cr
j		&ja		&BA		&0B1C		&j	&consonant ja	\cr
J		&jha		&BB		&0B1D		&jh	&consonant jha	\cr
F		&\~na		&BC		&0B1E		&\~{}n	&consonant nya	\cr\noalign{\medskip}
%
q		&\d{t}a		&BD		&0B1F		&.t	&consonant tta	\cr
Q		&\d{t}ha	&BE		&0B20		&.th	&consonant ttha	\cr
w		&\d{d}a		&BF		&0B21		&.d	&consonant dda	\cr
W		&\d{d}ha	&C0		&0B22		&.dh	&consonant ddha	\cr
N		&\d{n}a		&C1		&0B23		&.n	&consonant nna	\cr\noalign{\medskip}
%
t		&ta		&C2		&0B24		&t	&consonant ta	\cr
T		&tha		&C3		&0B25		&th	&consonant tha	\cr
d		&da		&C4		&0B26		&d	&consonant da	\cr
D		&dha		&C5		&0B27		&dh	&consonant dha	\cr
n		&na		&C6		&0B28		&n	&consonant na	\cr\noalign{\medskip}
%
p		&pa		&C8		&0B2A		&p	&consonant pa	\cr
P		&pha		&C9		&0B2B		&ph	&consonant pha	\cr
b		&ba		&CA		&0B2C		&b	&consonant ba	\cr
B		&bha		&CB		&0B2D		&bh	&consonant bha	\cr
m		&ma		&CC		&0B2E		&m	&consonant ma	\cr\noalign{\medskip}
%
Y		&ya		&CD		&0B5F		&y	&consonant ya	\cr
y		&\.ya		&CE		&0B2F		&.y	&consonant yya	\cr
r		&ra		&CF		&0B30		&r	&consonant ra	\cr
l		&la		&D1		&0B32		&l	&consonant la	\cr
v		&va		&D4		&0B13 0B4D 0B2C		&.v	&consonant va	\cr
\orbadot&ba		&--		&--		&.b		&consonant ba with dot \cr
z		&\'sa		&D5		&0B36		&sh	&consonant sha	\cr
S		&\d{s}a		&D6		&0B37		&.s	&consonant ssa	\cr
s		&sa		&D7		&0B38		&s	&consonant sa	\cr
h		&ha		&D8		&0B39		&h	&consonant ha	\cr
L		&\d{l}a		&D2		&0B33		&.l	&consonant lla	\cr\noalign{\medskip}
%
w\ornukta	&\d{r}a		&BF E9		&0B5C		&.r	&consonant rra	\cr
W\ornukta	&\d{r}ha	&C0 E9		&0B5D		&.rh	&consonant rrha	\cr\noalign{\medskip\leftline{\it vowel signs}\medskip}
%
\dotcircle A	&\=a		&DA		&0B3E		&aa	&vowel sign aa	\cr
\dotcircle[	&i		&DB		&0B3F		&i	&vowel sign i	\cr
\dotcircle X	&\=\i		&DC		&0B40		&ii	&vowel sign ii	\cr
\dotcircle]	&u		&DD		&0B41		&u	&vowel sign u	\cr
\dotcircle Z	&\=u		&DE		&0B42		&uu	&vowel sign uu	\cr
\dotcircle\orsignri&\d{r}		&DF		&0B43		&R	&vowel sign ri	\cr
\dotcircle\orsignrii&\d{\=r}		&DF E9		&--		&RR	&vowel sign rii	\cr
\dotcircle\orsignli&\d{l}		&DB E9		&--		&L	&vowel sign li	\cr
\dotcircle\orsignlii&\d{\=l}		&DC E9		&--		&LL	&vowel sign lii	\cr
<\dotcircle	&e		&E1		&0B47		&e	&vowel sign e	\cr
<\dotcircle>	&ai		&E2		&0B48		&ai	&vowel sign ai	\cr
<\dotcircle A	&o		&E4		&0B4B		&o	&vowel sign o	\cr
<\dotcircle*	&au		&E5		&0B4C		&au	&vowel sign au	\cr\noalign{\medskip\leftline{\it additional vowels}\medskip}
%
e\orsecya>	&\^e	&--	&0BOF 0B4D 0B5F	&\~{}e	\cr
o\orsecya>	&\^o	&--	&0B13 0B4D 0B5F	&\~{}o	\cr\noalign{\medskip\leftline{\it other signs and symbols}\medskip}
%
\dotcircle+	&		&E8		&0B4D		&+	&halant		\cr
\oravagraha	&		&EA E9		&0B3D		&.a	&avagraha	\cr
\organesh	&		&--		&0B70		&	&isshar		\cr
\oromsign	&		&A1 E9		&--		&.o	&om sign	\cr\noalign{\medskip\leftline{\it digits}\medskip}
%
0		&0		&F1		&0B66		&0	&digit zero	\cr
1		&1		&F2		&0b67		&1	&digit one	\cr
2		&2		&F3		&0B68		&2	&digit two	\cr
3		&3		&F4		&0B69		&3	&digit three	\cr
4		&4		&F5		&0B6A		&4	&digit four	\cr
5		&5		&F6		&0B6B		&5	&digit five	\cr
6		&6		&F7		&0B6C		&6	&digit six	\cr
7		&7		&F8		&0B6D		&7	&digit seven	\cr
8		&8		&F9		&0B6E		&8	&digit eight	\cr
9		&9		&FA		&0B6F		&9	&digit nine	\cr\noalign{\medskip\leftline{\it conjunct control}\medskip}
%
\orkra	&kra	&B3 E8 CF		&0B15 0B4D 0B30	&kra	&{\it ordinary conjunct}\cr
k+r		&kra	&B3 E8 E8 CF	&0B15 0B4D 200C 0B30	&k+ra	&{\it explicit halant}\cr
k\orsecra	&kra	&B3 E8 E8 E8 CF	&0B15 0B4D 200D 0B30	&k++ra	&{\it alternate conjunct}\cr
}

\vfill\eject
\leftline{\bf Table of Consonant-Vowel Sign Combinations}
\bigskip

Vowel-signs often combine with the consonant or conjunct they modify.

\medskip

\halign{\it#\quad\hfill&&\orXII#\quad\hfill\cr
	&\it a	&\it\=a &\it i	&\it\=\i &\it u	&\it\=u &\it\d{r} &\it e &\it ai &\it o &\it au &\it \.m&\it \d{m}&\it \d{h}\cr
	&a	&aA	&i	&I	&u	&U	&\orvowelri	 &e	&E	&o	&O	&a/	&aM	&aH	\cr\noalign{\smallskip}
k	&k	&kA	&k[	&kX	&k]	&kZ	&k\orsignri &<k	&<k>	&<kA	&<k*	&k/	&kM	&kH	\cr
kh	&K	&KA	&K[	&KX	&K]	&KZ	&K\orsignri &<K	&<K>	&<KA	&<K*	&K/	&KM	&KH	\cr
g	&g	&gA	&g[	&gX	&g]	&gZ	&g\orsignri &<g	&<g>	&<gA	&<g*	&g/	&gM	&gH	\cr
gh	&G	&GA	&G[	&GX	&G]	&GZ	&G\orsignri &<G	&<G>	&<GA	&<G*	&G/	&GM	&GH	\cr
\.n	&f	&fA	&f[	&fX	&f]	&fZ	&f\orsignri &<f	&<f>	&<fA	&<f*	&f/	&fM	&fH	\cr\noalign{\smallskip}
%
c	&c	&cA	&c[	&cX	&c]	&cZ	&c\orsignri &<c	&<c>	&<cA	&<c*	&c/	&cM	&cH	\cr
ch	&C	&CA	&C[	&CX	&C]	&CZ	&C\orsignri &<C	&<C>	&<CA	&<C*	&C/	&CM	&CH	\cr
j	&j	&jA	&j[	&jX	&j]	&jZ	&j\orsignri &<j	&<j>	&<jA	&<j*	&j/	&jM	&jH	\cr
jh	&J	&JA	&J[	&JX	&J]	&JZ	&J\orsignri &<J	&<J>	&<JA	&<J*	&J/	&JM	&JH	\cr
\~n	&F	&FA	&F[	&FX	&F]	&FZ	&F\orsignri &<F	&<F>	&<FA	&<F*	&F/	&FM	&FH	\cr\noalign{\smallskip}
%
\d{t}	&q	&qA	&q[	&qX	&q]	&qZ	&q\orsignri &<q	&<q>	&<qA	&<q*	&q/	&qM	&qH	\cr
\d{t}h	&Q	&QA	&Q[	&QX	&Q]	&QZ	&Q\orsignri &<Q	&<Q>	&<QA	&<Q*	&Q/	&QM	&QH	\cr
\d{d}	&w	&wA	&w[	&wX	&w]	&wZ	&w\orsignri &<w	&<w>	&<wA	&<w*	&w/	&wM	&wH	\cr
\d{d}h	&W	&WA	&W[	&WX	&W]	&WZ	&W\orsignri &<W	&<W>	&<WA	&<W*	&W/	&WM	&WH	\cr
\d{n}	&N	&NA	&N[	&NX	&N]	&NZ	&N\orsignri &<N	&<N>	&<NA	&<N*	&N/	&NM	&NH	\cr\noalign{\smallskip}
%
t	&t	&tA	&t[	&tX	&t]	&tZ	&t\orsignri &<t	&<t>	&<tA	&<t*	&t/	&tM	&tH	\cr
th	&T	&TA	&T[	&TX	&T]	&TZ	&T\orsignri &<T	&<T>	&<TA	&<T*	&T/	&TM	&TH	\cr
d	&d	&dA	&d[	&dX	&d]	&dZ	&d\orsignri &<d	&<d>	&<dA	&<d*	&d/	&dM	&dH	\cr
dh	&D	&DA	&D[	&DX	&D]	&DZ	&D\orsignri &<D	&<D>	&<DA	&<D*	&D/	&DM	&DH	\cr
n	&n	&nA	&n[	&nX	&n]	&nZ	&n\orsignri &<n	&<n>	&<nA	&<n*	&n/	&nM	&nH	\cr\noalign{\smallskip}
%
p	&p	&pA	&p[	&pX	&p]	&pZ	&p\orsignri &<p	&<p>	&<pA	&<p*	&p/	&pM	&pH	\cr
ph	&P	&PA	&P[	&PX	&P]	&PZ	&P\orsignri &<P	&<P>	&<PA	&<P*	&P/	&PM	&PH	\cr
b	&b	&bA	&b[	&bX	&b]	&bZ	&b\orsignri &<b	&<b>	&<bA	&<b*	&b/	&bM	&bH	\cr
bh	&B	&BA	&B[	&BX	&B]	&BZ	&B\orsignri &<B	&<B>	&<BA	&<B*	&B/	&BM	&BH	\cr
m	&m	&mA	&m[	&mX	&m]	&mZ	&m\orsignri &<m	&<m>	&<mA	&<m*	&m/	&mM	&mH	\cr\noalign{\smallskip}
%
y	&Y	&YA	&Y[	&YX	&Y]	&YZ	&Y\orsignri &<Y	&<Y>	&<YA	&<Y*	&Y/	&YM	&YH	\cr
\.{y}	&y	&yA	&y[	&yX	&y]	&yZ	&y\orsignri &<y	&<y>	&<yA	&<y*	&y/	&yM	&yH	\cr
r	&r	&rA	&r[	&rX	&r]	&rZ	&r\orsignri &<r	&<r>	&<rA	&<r*	&r/	&rM	&rH	\cr
l	&l	&lA	&l[	&lX	&l]	&lZ	&l\orsignri &<l	&<l>	&<lA	&<l*	&l/	&lM	&lH	\cr
%v	&v	&vA	&v[	&vX	&v]	&vZ	&v\orsignri &<v	&<v>	&<vA	&<v*	&v/	&vM	&vH	\cr
\'s	&z	&zA	&z[	&zX	&z]	&zZ	&z\orsignri &<z	&<z>	&<zA	&<z*	&z/	&zM	&zH	\cr
\d{s}	&S	&SA	&S[	&SX	&S]	&SZ	&S\orsignri &<S	&<S>	&<SA	&<S*	&S/	&SM	&SH	\cr
s	&s	&sA	&s[	&sX	&s]	&sZ	&s\orsignri &<s	&<s>	&<sA	&<s*	&s/	&sM	&sH	\cr
h	&h	&hA	&h[	&hX	&h]	&hZ	&h\orsignri &<h	&<h>	&<hA	&<h*	&h/	&hM	&hH	\cr
\d{l}	&L	&LA	&L[	&LX	&L]	&LZ	&L\orsignri &<L	&<L>	&<LA	&<L*	&L/	&LM	&LH	\cr\noalign{\smallskip}
	&\dotcircle 	&\dotcircle A	&\dotcircle [	
&\dotcircle X	&\dotcircle ]	&\dotcircle Z	&\dotcircle\orsignri
&<\dotcircle 	&<\dotcircle >	&<\dotcircle A	&<\dotcircle *
&\dotcircle / &\dotcircle M	&\dotcircle H	\cr
}

\bigskip
\leftline{\bf combinations of vowel signs, reph and candrabindu}
\medskip

When vowel signs, reph and candrabindu appear together, the following
ligatures are used.

\medskip

{\orXII
\dotcircle[\orcandrabindu\ %
\dotcircle[\orreph\ %
\dotcircle[\orreph\orcandrabindu\ %
<\dotcircle>\orcandrabindu\ %
<\dotcircle>\orreph\ %
<\dotcircle>\orreph\orcandrabindu\ %
<\dotcircle*\orcandrabindu\ %
<\dotcircle*\orreph\ %
<\dotcircle*\orreph\orcandrabindu\ %
}

\vfill\eject
\leftline{\bf Table of Conjunct Consonants}
\bigskip

The conjunct consonants in Oriya are very intricate and irregular. The table
below gives all the conjuncts included in the font, however, not all of these
will be automatically used by the pre-processor.

The conjuncts were collected from various sources. I consulted various Oriya
grammars and dictionaries, and inspected the Oriya font produced by C-DAC. All 
conjuncts encountered, except for those fully regular, like {\or k\orsecra} {\it kra}
and {\or p\orsecya} {\it pya,} have been included in this list.

\medskip

\begindoublecolumns

\halign{%
\orXII#\rm\ \hfill&#\  \hfill&
\orXII#\rm\ \hfill&#\  \hfill&
\orXII#\rm\ \hfill&#\  \hfill&
\orXII#\rm\ \hfill&#\  \hfill&
\orXII#\rm\quad\quad\hfill&#\cr
k	&+&k	&&	&&	&=&\orkka		&kka		\cr
k   &+&q    &&  &&  &=&\orkTa		&k\d{t}a	\cr
k	&+&t	&&	&&	&=&\orkta		&kta		\cr
k	&+&r	&&	&&	&=&\orkra {\rm\ or} k\orsecra	&kra	\cr
k	&+&l	&&	&&	&=&k\orsecla	&kla		\cr
k	&+&b	&&	&&	&=&k\orsecva	&kva		\cr
k	&+&S	&&	&&	&=&\orkSa		&k\d{s}a	\cr
k	&+&S	&+&N	&&	&=&\orkSNa	&k\d{s}\d{n}a	\cr
k	&+&s	&&	&&	&=&\orksa		&ksa		\cr
%
K	&+&Y	&&	&&	&=&K\orsecya	&khya		\cr
%
g	&+&g	&&	&&	&=&\orgga		&gga		\cr
g	&+&D	&&	&&	&=&\orgdha		&gdha		\cr
g	&+&n	&&	&&	&=&g\orsecna	&gna		\cr
g	&+&r	&&	&&	&=&g\orsecra	&gra		\cr
g	&+&l	&&	&&	&=&g\orsecla	&gla		\cr
%
G	&+&n	&&	&&	&=&G\orsecna	&ghna		\cr
%
f	&+&k	&&	&&	&=&\orngka		&\.nka		\cr
f	&+&K	&&	&&	&=&\orngkha		&\.nkha		\cr
f	&+&g	&&	&&	&=&\orngga		&\.nga		\cr
f	&+&G	&&	&&	&=&\ornggha		&\.ngha		\cr
%
c	&+&c	&&	&&	&=&\orcca		&cca		\cr
c	&+&C	&&	&&	&=&\orccha		&ccha		\cr
%
j	&+&j	&&	&&	&=&\orjja		&jja		\cr
j	&+&J	&&	&&	&=&\orjjha		&jjha		\cr
j	&+&F	&&	&&	&=&\orjnya		&j\~na		\cr
j	&+&Y	&&	&&	&=&j\orsecya	&jya		\cr
j	&+&b	&&	&&	&=&j\orsecva	&jva		\cr
%
F	&+&c	&&	&&	&=&\ornyca		&\~nca		\cr
F	&+&C	&&	&&	&=&\ornyca\orseccha	&\~ncha		\cr
F	&+&j	&&	&&	&=&\ornyja		&\~nja		\cr
F	&+&J	&&	&&	&=&\ornyjha		&\~njha		\cr
%
q	&+&q	&&	&&	&=&\orTTa		&\d{t}\d{t}a	\cr
%
w   &+&g	&&	&&	&=&\orDga		&\d{d}ga		\cr
w\ornukta&+&g&&	&&	&=&\orRga		&\d{r}ga		\cr
w	&+&w	&&	&&	&=&\orDDa		&\d{d}\d{d}a	\cr
%
N	&+&q	&&	&&	&=&\orNTa		&\d{n}\d{t}a	\cr
N	&+&Q	&&	&&	&=&\orNTha		&\d{n}\d{t}ha	\cr
N	&+&w	&&	&&	&=&\orNDa		&\d{n}\d{d}a	\cr		
N	&+&W	&&	&&	&=&\orNDha		&\d{n}\d{d}ha	\cr
N	&+&N	&&	&&	&=&\orNNa		&\d{n}\d{n}a	\cr
%
t	&+&t	&&	&&	&=&\ortta		&tta		\cr
t	&+&n	&&	&&	&=&\ortna		&tna		\cr
t   &+&p	&&	&&	&=&\ortpa		&tpa		\cr
t	&+&m	&&	&&	&=&\ortma {\rm\ or} t\orsecma	&tma	\cr
t	&+&r	&&	&&	&=&\ortra {\rm\ or} t\orsecra	&tra	\cr
t	&+&s	&&	&&	&=&\ortsa		&tsa		\cr
%
d	&+&g	&&	&&	&=&\ordga		&dga		\cr
d	&+&d	&&	&&	&=&\ordda		&dda		\cr
d	&+&D	&&	&&	&=&\orddha		&ddha		\cr
d	&+&B	&&	&&	&=&\ordbha		&dbha		\cr
d	&+&m	&&	&&	&=&d\orsecma	&dma		\cr
%
D	&+&Y	&&	&&	&=&\ordhya		&dhya		\cr
D	&+&b	&&	&&	&=&D\orsecva	&dhva		\cr
%
n	&+&t	&&	&&	&=&\ornta		&nta		\cr
n	&+&t	&+&Y	&&	&=&\ornta\orsecya	&ntya		\cr
n	&+&t	&+&r	&&	&=&\orntra	&ntra		\cr
n	&+&T	&&	&&	&=&n\orsectha	&ntha		\cr
n	&+&d	&&	&&	&=&\ornda		&nda		\cr
n	&+&D	&&	&&	&=&\orndha		&ndha		\cr
n	&+&n	&&	&&	&=&n\orsecna	&nna		\cr
%
p	&+&Y	&&	&&	&=&p\orsecya	&pya		\cr
%
b	&+&j	&&	&&	&=&\orbja		&bja		\cr
b	&+&d	&&	&&	&=&\orbda		&bda		\cr
b   &+&D	&&	&&	&=&\orbdha		&bdha		\cr
b	&+&b	&&	&&	&=&\orbba		&bba		\cr
%
p	&+&t	&&	&&	&=&\orpta		&pta		\cr
p	&+&s	&&	&&	&=&\orpsa		&psa		\cr
%
m	&+&p	&&	&&	&=&\ormpa		&mpa		\cr
m	&+&P	&&	&&	&=&\ormpha		&mpha		\cr
m	&+&b	&&	&&	&=&m\orsecva	&mba		\cr
m	&+&B	&&	&&	&=&m\orsecbha	&mbha		\cr
m	&+&m	&&	&&	&=&\ormma		&mma		\cr
%
r	&+&N	&+&N	&&	&=&\orNNa\orreph	&r\d{n}\d{n}a	\cr
r	&+&b	&&	&&	&=&b\orreph		&rba		\cr
%
l	&+&k	&&	&&	&=&\orlka		&lka		\cr
l	&+&p	&&	&&	&=&\orlpa		&lpa		\cr
l	&+&P	&&	&&	&=&\orlpha		&lpha		\cr
l	&+&l	&&	&&	&=&\orlla		&lla		\cr
%
z	&+&c	&&	&&	&=&\orshca		&\'sca		\cr
z	&+&C	&&	&&	&=&z\orseccha	&\'scha		\cr
z	&+&q	&&	&&	&=&\orshTa		&\'s\d{t}a	\cr
z	&+&n	&&	&&	&=&z\orsecna	&\'sna		\cr
%
S	&+&k	&&	&&	&=&\orSka		&\d{s}ka	\cr
S	&+&q	&&	&&	&=&\orSTa		&\d{s}\d{t}a	\cr
S	&+&Q	&&	&&	&=&\orSTha		&\d{s}\d{t}ha	\cr
S   &+&N    &&  &&  &=&\orSNa		&\d{s}\d{n}a	\cr
S	&+&p	&&	&&	&=&\orSpa		&\d{s}pa	\cr
S	&+&P	&&	&&	&=&\orSpha		&\d{s}pha	\cr
%
s	&+&k	&&	&&	&=&\orska		&ska		\cr
s	&+&K	&&	&&	&=&\orskha		&skha		\cr
s	&+&t	&&	&&	&=&\orsta		&sta		\cr
s	&+&t	&+&r	&&	&=&\orstra	&stra		\cr
s	&+&T	&&	&&	&=&s\orsectha	&stha		\cr
s	&+&p	&&	&&	&=&\orspa		&spa		\cr
s	&+&P	&&	&&	&=&\orspha		&spha		\cr
%
h	&+&n	&&	&&	&=&\orhna		&hna		\cr
h	&+&m	&&	&&	&=&\orhma		&hma		\cr
h	&+&b	&&	&&	&=&\orhva		&hva		\cr
}

\enddoublecolumns

\vfill\eject

% Some conjunct-vowel sign combinations
% 
% {\or k] \orkka] \ormma Z \orpsa\orsecra \orpsa] p] s]  k\orsecva]}

\beginsection Glyph Chart

This chart gives all the glyphs and their positions in the Oriya font. Software should
not rely on a certain glyph having a certain position in this chart, but use the
symbolic names for glyphs and the mapping tables supplied with the font. The font
includes several copies of the same glyph in different positions. Although such
copies are not strictly necessary for a composing system as strong as \TeX, they will make
life much easier to use the font in a Windows environment, to which I intend
to adapt it.

\bigskip

\input chart

\table or12

\bye
