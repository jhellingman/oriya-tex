% 2kol.tex
% last edit: 20-NOV-1991 --JH

%
% Name:     two_col.tex                 Double Column Format for plain TeX.
%
% Author:
%   E. Myers and F.E. Paige
%   clean up and modifications by Jeroen Hellingman
%
% Description:
%
%   two_col.tex produces double-column output with balanced columns.
%   Adapted from The TeXBook, pg 417, and The TUGboat, Vol 6, pg 29.
%
%   \setdoublecolumns initializes the double column format, including
%   setting \hsize and \vsize:
%
%       \setdoublecolumns#1#2#3         initializes double column format
%                                           #1 = overall width
%                                           #2 = overall height
%                                           #3 = column width
%
%   \begindoublecolumns and \enddoublecolumns are used to switch between
%   single and double column format:
%
%       \begindoublecolumns             starts double column format
%       \enddoublecolumns               returns to single column format
%                                       and balances columns
%       \leftcolrule                    draws a rule across left column
%       \rightcolrule                   draws a rule across right column
%
%
%   A typical TeX double-column document will contain:
%
%   \setdoublecolumns{width}{height}{colwidth}  % in cm, in, pt,...
%   ... title page, etc ...
%   ... title, author, etc ...                  % default is single column
%   \begindoublecolumns                         % switch to 2 columns
%   ... text ...
%   \enddoublecolumns                           % switch to 1 column
%   \leftcolrule                                % rule left column (optional)
%   $$
%   ... long equation, table, etc ...           % across both columns
%   $$
%   \rightcolrule                               % rule right column (optional)
%   \begindoublecolumns                         % again 2 columns 
%   ... text ...
%   \enddoublecolumns                           % end
%
%   Since double columns are narrow, the normal TeX spacing will produce
%   many overfull hboxes. To avoid this, the interword skip \spaceskip is
%   replaced by \doublecolskip with the default value
%
%       \doublecolskip=.3333em plus .2em minus .1em
%
%   and the \hyphenpenalty is set to zero. To balance the columns, it is 
%   necessary to allow some vertical stretch; for example use
%
%       \parskip=\smallskipamount
%
%   \topinsert and \midinsert, etc. will usually work with double column
%   format. However, because the macros build a box longer than two columns and 
%   then split it, it is possible for a \midinsert to fit on the oversize
%   page but not on the real one. In this case it will be lost. If this
%   happens, move the insertion to another position.
%
%   To type oversize documents on an 8.5 x 11 inch laser printer, use
%   \tenpoint and set the dimensions in \setdoublecolumns to 10/12 of the
%   desired size. Then the document enlarged by 1.2 will be correct.
%
%%%%%%%%%%%%%%%%%%%%%%%%%%%%%%%%%%%%%%%%%%%%%%%%%%%%%%%%%%%%%%%%%%%%%%%%%%

\message{Double Column Format.}

\newdimen\colwidth \newdimen\bigcolheight \newdimen\leftpluscenter
\newdimen\pagewidth \newdimen\pageheight \newdimen\ruleht
\ruleht=.5pt
         
\newskip\doublecolskip \newskip\savespaceskip
%\doublecolskip=.3333em plus .3333em minus .1em
\doublecolskip=\spaceskip % changed --JH, use \emergencystretch

\newbox\partialpage
\newdimen\savesize

%%%%%%%%%%%%%%%%%%%%%%%%%%%%%%%%%%%%%%%%%%%%%%%%%%%%%%%%%%%%%%%%%%%%%%%%%%
%   From the TeXBook:

\newif\iftitle
\newinsert\margin
\dimen\margin=\maxdimen             % no limit on the number of marginal notes
%\def\rhead{}                        % \rhead contains the running headline
%\def\leftheadline{\hbox to \pagewidth{%
%    \vbox to 10pt{}                 % strut to position the baseline
%    \llap{\tenbf\folio\kern1pc}     % folio to left of text
%    \tenit\rhead\hfil}}             % running hed flush left
%\def\rightheadline{\hbox to \pagewidth{\vbox to 10pt{}%
%    \hfil\tenit\rhead\/             % running head flush right
%    \rlap{\kern1pc\tenbf\folio}}}   % folio to right of text
% No nonsens headline --JH:
\headline={\hfill\folio\hfill}
\newtoks\leftheader   \leftheader=\headline
\newtoks\rightheader  \rightheader=\headline
\def\leftheadline{\hbox to \pagewidth{\vbox to 10pt{}%
    \the\leftheader}}
\def\rightheadline{\hbox to \pagewidth{\vbox to 10pt{}%
    \the\rightheader}}
% --JH.

\def\onepageout#1{\shipout\vbox{    % here we define one page of output
    \offinterlineskip               % butt the boxes together
    \vbox to 3pc{                   % this part goes on top of the 44pc pages
      \iftitle \global\titlefalse \setcornerrules
      \else\ifodd\pageno\rightheadline\else\leftheadline\fi\fi \vfill}
    \vbox to \pageheight{
      \ifvoid\margin\else           % marginal info is present
       \rlap{\kern31pc\vbox to0pt{\kern4pt\box\margin \vss}}\fi
      #1                            % now insert the main information
      \ifvoid\footins\else          % footnote info present
       \vskip\skip\footins \kern-3pt
       \hrule height\ruleht width\pagewidth \kern-\ruleht \kern3pt
       \unvbox\footins\fi
      \boxmaxdepth=\maxdepth}}
  \advancepageno}

%%%%%%%%%%%%%%%%%%%%%%%%%%%%%%%%%%%%%%%%%%%%%%%%%%%%%%%%%%%%%%%%%%%%%%%%%%
%   From the TugBoat, with some additions. Note that calling
%   \setdoublecolumns redefines \output.

\def\setdoublecolumns#1#2#3{
        \global\hsize=#1 \global\vsize=#2 \global\colwidth=#3
        \pagewidth=\hsize \pageheight=\vsize
        \bigcolheight=\vsize 
        \multiply \bigcolheight by 2 \advance \bigcolheight by 1cm
        \leftpluscenter=\pagewidth \advance \leftpluscenter by -\colwidth
        \output{\onepageout{\unvbox255}}}

\def\begindoublecolumns{\begingroup
        \savesize=\vsize
        \output={\global\setbox\partialpage=\vbox{\unvbox255}}\eject
        \output={\doublecolumnout} \hsize=\colwidth \vsize=\bigcolheight
        \advance \vsize by -2\ht\partialpage
        \global\savespaceskip=\spaceskip \global\spaceskip=\doublecolskip
        \global\displaywidth=\colwidth
        \global\hyphenpenalty=0}

\def\doublecolumns{\begindoublecolumns}         % synonym

\def\enddoublecolumns{\output={\balancecolumns}\eject
        \global\output={\onepageout{\unvbox255}}
        \global\vsize=\savesize
        \endgroup \pagegoal=\vsize
        \global\spaceskip=\savespaceskip \global\displaywidth=\hsize
        \global\hyphenpenalty=50}
                                      
\def\doublecolumnout{\dimen0=\pageheight
        \advance \dimen0 by -\ht\partialpage \splittopskip=\topskip
        \setbox0=\vsplit255 to \dimen0
        \setbox2=\vsplit255 to \dimen0
        \onepageout\pagesofar
        \global\vsize=\bigcolheight
        \unvbox255 \penalty\outputpenalty}

\def\pagesofar{\unvbox\partialpage
        \wd0=\hsize \wd2=\hsize \hbox to \pagewidth{\box0\hfil\box2}}

\def\balancecolumns{\setbox0=\vbox{\unvbox255} \dimen0=\ht0
        \advance \dimen0 by \topskip \advance \dimen0 by -\baselineskip
        \divide \dimen0 by 2 \splittopskip=\topskip
        {\vbadness=10000 \loop \global\setbox3=\copy0
        \global\setbox1=\vsplit3 to \dimen0
        \ifdim \ht3 > \dimen0 \global\advance \dimen0 by 1pt \repeat}
        \setbox0=\vbox to \dimen0{\unvbox1}
        \setbox2=\vbox to \dimen0{\unvbox3}
        \global\output={\balancingerror}
        \pagesofar}

\newhelp\balerrhelp{Please change the page into one that works.}

\def\balancingerror{\errhelp=\balerrhelp
        \errmessage{Page cannot be balanced}
        \onepageout{\unvbox255}}

\def\leftcolrule{\vskip 2pt \nointerlineskip 
        \vbox{\hrule width\colwidth height\ruleht} 
        \nointerlineskip \vskip 4pt \relax}

\def\rightcolrule{\vskip 4pt \nointerlineskip 
        \moveright\leftpluscenter \vbox{\hrule width\colwidth height\ruleht}
        \vskip 2pt \relax \nointerlineskip}

% end of two_col.tex
